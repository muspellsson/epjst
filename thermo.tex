\documentclass[epjST]{svjour}
%
\usepackage{graphics}
%
\begin{document}
%
\title{Finite-time thermodynamics: problems, methods, results}
\author{A.M. Tsirlin\inst{1}\fnmsep\thanks{\email{tsirlin@sarc.botik.ru}} \and I.A. Sukin\inst{2}\fnmsep\thanks{\email{muspellsson@undeadbsd.org}} }
%
\institute{Program Systems Institute of RAS, Pereslavl-Zalessky,  Russia}
%
\abstract{
Insert your abstract here.
} %end of abstract
%
\maketitle
%
\section{Introduction}
\label{intro}
The development of thermodynamics, beginning from the Carnot’s work, is closely linked with extremal problems considering limiting capabilities of thermodynamical systems. If Carnot had posed the heat engine’s maximum COP (coefficient of performance) problem mathematically strict: ''To find such a law $T(t)$ of the working body’s temperature changing, that a ratio $\rho$ of the obtained work to the heat taken from the hot source would be maximum, where the heat engine takes the heat from the source with the temperature $T_+$ and gives it to the source with the temperature $T_-$'', he would not have got the solution of this problem. That is because the duration of the cycle $\tau$ is one of the sought-for variables in the given problem. The set of admissible values for this variable is bounded only by the non-negativity condition and therefore this set is not closed. Hence, according to the Weierstrass theorem (which Carnot could not know about) the problem may have no solutions. Really, the maximum of $\rho$ does not exist. The upper bound of the COP (supremum) is reached in a limit as $\tau$ approaches the infinity.

This characteristic turned to be typical for many others extremal thermodynamical problems (problem of minimal mixture separation work, problem of maximum COP for systems with sources of the finite capacity). The classical solution of these problems gives reversible processes in which exchange flows are close to zero, so the exchange of the finite amount of a matter or an energy takes an infinite amount of time.

We must note that the Carnot's heat engine may have the finite power output, but only in the case when coefficients of the heat exchange between heat sources and the working body of the heat engine are arbitrarily large. In this case the Carnot COP is the ratio between the ''reversible power output'' and the flow of the heat taken from the hot source.
\subsection*{The history of the finite-time thermodynamics}
''The problem of the maximum power output'' seems to be the first problem of the finite-time thermodynamics. This problem considers such form of the heat engine cycle that engine's power output would be maximum\cite{Nov}\cite{CurAhl} and others.

This problem was very actual for the newborn atomic energetics in 1950s, because the cost price of nuclear power plants were high and the fuel cost was relatively small. Getting the maximum power output was far more important than getting the maximum COP in these conditions.

The problem of the maximum power output considers heat and power flows instead of the heat amount and the work. The heat exchange kinetics, coefficients of the heat exchange between the working body and sources are also taken into account. Authors of the above-mentioned books and many other studies often solved this problem independently but using the same scheme:

\begin{enumerate}
\item Flows of the heat exchange for the contact with every source were assumed to be proportional to the difference of the source and the working body temperatures (Newtonian kinetics).
\item The sought-for cycle assumed to be analogous the Carnot cycle: it consists of two isotherms and two adiabats, and temperatures of the working body contact with sources were selected using the condition of maximum power output in respect with the energy conservation law and the fact that the flow of the entropy coming from the hot source must be equal to the flow of the entropy given to the cold source. The latter condition means that processes inside of the working body were assumed to be reversible.

It turned up to be that the maximum power output of the heat engine, contacting with reservoirs having temperatures $T_+$ and $T_-$ is
\begin{equation}
N_{max} = \frac{\alpha_1\alpha_2}{\alpha_1+\alpha_2}\left(\sqrt{T_+}-\sqrt{T_-}\right)^2,
\end{equation}
where $\alpha_1$ and $\alpha_2$ are the coefficients of the heat exchange for the contact with sources. The COP of the cycle for the maximum power output does not depend on the heat exchange coefficients and is equal to
\begin{equation}
\eta = 1 - \sqrt{\frac{T_-}{T_+}}
\end{equation}

\end{enumerate}

These works did not answer the following questions:
\begin{enumerate}
\item Which kinetics of the heat exchange gives the cycle of the maximum power output, consisting of two isotherms and two adiabats?
\item Is the COP of a such cycle always independent of the heat exchange coefficients?
\item What is the form of the heat engine cycle, maximizing the COP for the given power output.
\end{enumerate}

These questions were answered in \cite{RosTs}\cite{BeKaSi} using the techniques of the average optimization developed in \cite{Ts1}.
 
It turned up that
\begin{enumerate}
\item The cycle consists of two isotherms and two adiabats for every kinetics satisfying the condition of the equality of the heat flow direction and the sign of the difference between temperatures of contacting bodies.
\item The COP corresponding to such cycle depends on kinetics coefficients, in the most general case.
\item The cycle of the heat engine with the maximum COP for the given power output consists of no more than three isotherms and three adiabats. The condition for which the number of isotherms is exactly two was also obtained.
\end{enumerate}

The development of the finite-time thermodynamics from the ''maximum power output problem'' to it's modern condition had begun after the work of Curzon and Ahlborn\cite{CurAhl}. Talented researchers P. Salamon, B. Andresen, K. H. Hoffman worked with them in the department which head was R. S. Berry. They realized that in real thermodynamical, heat and mass exchange, chemical processes the limited duration of the process has very important role. It means we must consider so-called ''irreversible processes'' having additional condition of the finiteness of the time. This leads us to a new field of thermodynamics --- the finite-time thermodynamics\cite{SalHoSch}\cite{AndSalBe}\cite{AndSalBe2}\cite{And}.

We think this term is not so well-chosen, we will see it in the typical problems overview in the next section. We will note that there is an important class of processes which are stationary in the time and distributed over the coordinate (steam turbine, stationary heat and mass exchange, processes with circulating working body and others). maximum capabilities of these processes are found using general techniques based on the thermodynamics-specific optimization and optimal control. These techniques in total form the field of the ''optimization thermodynamics'' (the term is proposed by L.I. Rozonoer).

In the following sections we will take a look on the typical problems, general techniques of its solving and, finally, their solutions.

\section{Formulations of the optimization thermodynamics problems}

Every non-isolated thermodynamical system exchanges energy and matter flows with the environment. These flows can be stationary or periodic in the stationary mode. In the latter case we will name the average flow intensity over some period simply the ''intensity''. The way the system functions (kinetics of heat and mass exchange, chemical reactions, and others) determines the relation between incoming and outgoing flows. On of the outgoing flows will be named the target flow. It's intensity is the productivity of the system. Incoming flows form the costs flow. For example, for the heat engine the target flow is the power output and the costs flow is the heat taken from the hot source.

There are some formulations of the typical optimization thermodynamics problems:

\paragraph{Problem 1.} \textit{The problem of the maximum power output} for the thermodynamical system of an arbitrary nature under some conditions. This is also the problem of the costs flow value at the maximum power output. This is the straight generalization of the maximum power output problem.

Here comes the question: for which systems power output is upper-bounded and for which ones we can infinitely enlarge the costs flow thereby infinitely enlarging the power output (it means that Problem 1 has no solution)?

\paragraph{Problem 2.} We have a system of two or more thermodynamical reservoirs and the working body. The working body is contacting with every reservoir in the stationary mode or alternately with each one. The working body produces the target flow. How the contacts of the working body must be organized in order to the maximum value of the target flow? What value must be considered the COP of such system?

\paragraph{Problem 3.} If we take the finite capacity sources instead of reservoirs in the Problem 2, what will be changed? In particular, what is the maximum work value that could be produced in the closed thermodynamical system for the fixed time amount. This is the problem of calculating the exergy of the system for the case when the process duration is not bounded.

\paragraph{Problem 4.} How one must organize the thermodynamical processes in order to get the minimum entropy producion for the given average intensity of the process (processes of minimum dissipation).

\paragraph{In particular,} how can we estimate the heat exchange process? How the heat exchange process of two vector flows must be organized in order to get the minimum entropy production for the given heat capacity and the total heat exchange coefficient?

\paragraph{Problem 5.} To plot the region of attainable modes of the thermodynamical system. The axes values are intensities of flows.

The solution of Problem 5 shows requirements to the system and how these requirements are linked to the restrictions of kinetics coefficients, process duration and others.

Given formulations does not exhaust the overall problem space of the optimization thermodynamics, but allows us to make assumptions about it's nature and the practical orientation of the problems arising. We will note that some of the problems above does not include the duration of the process as a variable.

\section{General techniques of solving the optimization thermodynamics problems, dissipation}

It is well-known that thermodynamical systems are characterized by two types of variables: intensive and extensive ones. The former ones does not change after merging of subsystems if they were equal in each subsystem before merging (temperatures, pressures, concentrations), the latter are added up after such merging (volume, the number of moles, internal energy).

In passive subsystems intensive variables are determined by the extensive ones and the equation of state. In active subsystems intensive variables are selected in order to reach some goal (intenvise variables are the control).

Such problems can be solved using the following scheme.

The first step in the analysis of the maximum capabilites of thermodynamical systems if the writing of the balance relations for the matter, energy and entropy. The relation for the entropy will include the summand characterizing the irreversibility of thermodynamical processes, --- the entropy production $\sigma$. This summand is equal to zero if every process in the system are reversible, and is above zero for irreversible processes. The non-negativity condition for the dissipation determines some attainability region in the space of incoming and outgoing flows. If there were additional conditions of the finite time or the average process intensity one could find the minimum dissipation value for these conditions. In any real system $\sigma \geq \sigma_{min}$, which constricts the attainability region. The attainability region takes into account the kinetics of processes and the size of the engine, through the coefficients of heat and mass transfer.

The second step is the deducing of the relation between some capability value and the dissipation $\sigma$. Natural capability values are usually monotonously decreasing with the increase of $\sigma$ and reach their maximum values in the reversible process. This leads to the estimates analogous to the Carnot engine COP for processes of the very different nature.

The third and the most difficult step is the solution of the problem of such processes organization for which the dissipation reaches it's minimum for the given restrictions.

In the complex system the total dissipation additively depends on the dissipation of each elementary process. So the important stage of the analysis is the finding of minimum dissipation conditions. The optimal processes organization in the complex system are reduced to the coordination of separate minimum dissipation processes.

\subsection{Thermodynamical balance relations}

Thermodynamical balance relations are the system of equations of the matter, energy and entropy balances. For the simplicity we will look at these equations for the open system. The left part of thermodynamical balance equations is zero in the stationary mode. Some flows are incoming to the system from the environment and some ones are created inside the system. One of these created flows is the entropy production, which is always non-negative. This turns the entropy balance equation into the inequality. Together with other equations it allocates the attainability region in the flow space. The bound of this region corresponds to reversible processes. But if we can solve the problem of finding the minimum entropy production $\sigma_{min}$ then balance equations with the condition $\sigma \geq \sigma_{min}$ allocates the region which bound corresponds to the processes of minimum dissipation. This region is included into the region bounded by reversible processes. Now we will examine it more closely.

\paragraph{Open system.} Thermodynamical balance equations determines relations between flows of each substance, energy and entropy, which are exchanged within the system and with environment. They also helps to figure out the relation between flow production and the speed of their changing. We will sum up all the flows, considering incoming flows as positive and outgoing ones as negative. Distinguishing flows as convective and diffusive we will denote latter ones with the index $d$. Unlike the convective flow the diffusive one depends on the difference between intensive variables of the system studied at the point where this flows comes in or goes out and intensive variables of the environment. We will use following denotations: $j$ --- flow index, $c_j$, $v_j$ --- internal energy of the flow and it's molar volume, $P_j$ --- pressure of the flow, $h_j = e_j + P_j v_j$ --- molar enthalpy, $h_{dj}$ --- enthalpy of the diffusive flow, $q_j$ --- heat flow, $N_a$ --- power output of the system.

The general form of balance equations is
\begin{equation}
\frac{dE}{dt} = \sum_j{g_j h_j} + \sum_j{q_{dj}} + \sum_j{q_j} - N_a,
\label{BalMat}
\end{equation}
\begin{equation}
\frac{dN_i}{dt} = \sum_j{g_j x_{ij}} + \sum_j{g_{dj} x_{dj}} + \sum_{\nu} {\alpha_{i\nu} W_{\nu}},
\label{BalEne}
\end{equation}
\begin{equation}
\frac{dS}{dt} = \sum_j{g_j s_j} + \sum_j{\frac{q_{dj} - \sum_i{g_{dj} \mu_{dij}}}{T_{dj}}} + \sum_{i\nu}{\frac{\mu_{i\nu}n_{i\nu}}{T_{\nu}}} + \sum_j{\frac{q_j}{T_j}} + \sigma.
\label{BalEnt}
\end{equation}
Where $n_{i\nu} = -\alpha_{i\nu}W_{\nu}$ is the intensity of the $i$-th substance production in the $\nu$-th reaction, $T_{\nu}$ is the temperature in the $\nu$-th reaction. If there is no diffusive flows:
\begin{equation}
\frac{dE}{dt} = \sum_j{g_j h_j} + \sum_j{q_j} - N_a,
\label{BalMatNoD}
\end{equation}
\begin{equation}
\frac{dN_i}{dt} = \sum_j {g_j x_{ij}} + \sum_{\nu}{\alpha_{i\nu}W_{\nu}},
\label{BalEneNoD}
\end{equation}
\begin{equation}
\frac{dS}{dt} = \sum_j{g_j s_j} + \sum_j{\frac{q_j}{T_j}} + \sum_{i\nu}{\frac{\mu_{i\nu}n_{i\nu}}{T_{\nu}}} + \sigma,
\label{BalEntNoD}
\end{equation}
where heat flows produced or consumed during chemical reactions are among other heat flows. These flows depends on the reaction speed.

Assuming the process is in the stationary mode, where $\frac{dE}{dt} = \frac{dN_i}{dt} = \frac{dS}{dt} = 0$, one can rewrite the differential relations of thermodynamical balances as simple finite relations. Balances can be written not for every moment of time but in average for the overall time, if the process is cyclic. Because states of the system are identical at the start and at the end of every cycle, overall variation of the energy, the substances amount and the entropy is zero. Balances in this case are reduced to the system of relations bounding average values of right parts of equations.

For closed systems, consisting of several equilibrium subsystems thermodynamical balances are
\[
\dot{E_0} = \sum_i{\dot{E_i}},
\]
\[
\dot{N_0} = \sum_i{\dot{N_i}},
\]
\[
\dot{S_0} = \sum_i{\dot{S_i}},
\]
where $i$ is the number of the subsystem and the index zero corresponds to the system at a whole. $\dot{E_i}$, $\dot{N_i}$ and $\dot{S_i}$ are determined by (\ref{BalMat})--(\ref{BalEnt}).

Fig.\ref{fig1} show flow schemes for some thermodynamical systems, and the table \ref{tab1} shows the relation between the capability value and the entropy production. It is easy to see, that with the increase of $\sigma$ the capability is decreased.

\section{Some results}

\subsection*{Minimum dissipation processes}
%\section{Section title}
%\label{sec:1}
%and \cite{RefJ}
%\subsection{Subsection title}
%\label{sec:2}
%as required. Don't forget to give each section
%and subsection a unique label (see Sect.~\ref{sec:1}).
%%
%
%\begin{figure}
%% Use the relevant command for your figure-insertion program
%% to insert the figure file.
%% For example, with the option graphics use
%%\resizebox{0.75\columnwidth}{!}{%
%%  \includegraphics{fig1.eps} }
%\caption{Please write your figure caption here.}
%\label{fig:1}       % Give a unique label
%\end{figure}
%%
%% For tables use
%\begin{table}
%\caption{Please write your table caption here.}
%\label{tab:1}       % Give a unique label
%% For LaTeX tables use
%\begin{tabular}{lll}
%\hline\noalign{\smallskip}
%first & second & third  \\
%\noalign{\smallskip}\hline\noalign{\smallskip}
%number & number & number \\
%number & number & number \\
%\noalign{\smallskip}\hline
%\end{tabular}
%\end{table}
%
\begin{thebibliography}{}
% and use \bibitem to create references.
\bibitem{Nov}
Novikov I.I., \textit{The efficiency of atomic power stations}, J. Nuclear Energy, \textbf{II 7}, (1958) 25--128
\bibitem{CurAhl}
Curzon F.L., Ahlburn B., \textit{Efficiency of a Carnot engine at maximum power output}, Amer. J. Physics, \textbf{43}, (1975) 22--24
\bibitem{RosTs}
Rozonoer L.I., Tsirlin A.M., \textit{Optimal control of thermodynamical systems}, Automatics and Telemechanics, \textbf{1, 2, 3}, (1983), in Russian
\bibitem{BeKaSi}
Berry R.S., Kazakov V.A., Sieniutycz S., Szwast Z., Tsirlin A.M., \textit{Thermodynamic Optimization of Finite Time Processes}, (Wiley, Chichester 1999)
\bibitem{Ts1}
Tsirlin A.M., \textit{Techniques of the average optimization and its applications}, (Moscow, Fizmatlit 1997), in Russian
\bibitem{SalHoSch}
Salamon P., Hoffman K.H., Schubert S., Berry R.S., Andresen B., \textit{What conditions make minimum entropy production equivalent to maximum power production?}, J. Non-Equilibrium Thermodyn., \textbf{26}, (2001)
\bibitem{AndSalBe}
Andresen B., Salamon P., Berry R.S. \textit{Thermodynamics in finite time}, Phys. Today, \textbf{62}, (1984)
\bibitem{AndSalBe2}
Andresen B., Salamon P., Berry R.S., \textit{Thermodynamics in finite time: extremals for imperfect heat engines}, J. Chem. Phys., \textbf{64 4}, (1977), 1571--1577
\bibitem{And}
Andresen B., \textit{Finite-time thermodynamics}, (Copenhagen, 1983)
\bibitem{TsMir}
Tsirlin A.M., Mironova V.A., Amelkin S.A., Kazakov V.A., \textit{Finite-time thermodynamics: Conditions of minimal dissipation for thermodynamic process with given rate}, Physical Review, \textbf{58 1}, (1998)
\bibitem{TsKaz}
Tsirlin A.M., Kazakov V.A., \textit{Maximal work problem in finite-time thermodynamics}, Physical Review, \textbf{1}, (2000)
\bibitem{TsGrig}
Tsirlin A.M., Grigorevsky I.N., \textit{Thermodynamical estimation of the limit capacity of irreversible binary distillation}, J. Non-Equilibrium Thermodynamics, \textbf{35}, (2010), 213--233
\bibitem{Ts2}
Tsirlin A.M., \textit{Mathematical models and optimal processes in macrosystems}, (Moscow, Science 2006), in Russian

\end{thebibliography}

\end{document}

% end of file template.tex

